\begin{frame}{Guardado rápido}
  \begin{columns}[onlytextwidth]
    \column{0.3\textwidth}
    \includegraphics[scale=0.9]{images/gitstash}
    \column{0.7\textwidth}
    \begin{itemize}
      \item \alert{Guardar los cambios}
        \mint{console}| $ git stash|
        \mint{console}| $ git stash save 'mensaje'|
      \item \alert{Guardar sólo los cambios preparados}
          \mint{console}| $ git stash --keep-index|
      \item \alert{Guardar de archivos no rastreados}
          \mint{console}| $ git stash -u|
      \item \alert{Guardar de forma selectiva}
          \mint{console}| $ git stash --patch|
    \end{itemize}
  \end{columns}
\end{frame}

\begin{frame}{Guardado rápido}
  \begin{columns}[onlytextwidth]
    \column{0.3\textwidth}
    \includegraphics[scale=0.9]{images/gitstash}
    \column{0.7\textwidth}
    \begin{itemize}
      \item \alert{Listarlos}
          \mint{console}| $ git stash list|
      \item \alert{Aplicar un guardado}
          \mint{console}| $ git stash apply <stash@{x}>|
          {\scriptsize \ \ \texttt{--index}: recupera el estado}
      \item \alert{Aplicar y sacar un guardado}
          \mint{console}| $ git stash pop <stash@{x}>|
      \item \alert{Crear una rama}
          \mint{console}| $ git stash branch <rama> <stash@{x}>|
    \end{itemize}
    {\scriptsize Si no se especifica \texttt{<stash@{x}>}, Git selecciona el más reciente.}
  \end{columns}
\end{frame}

\begin{frame}{Guardado rápido}
  \begin{columns}[onlytextwidth]
    \column{0.3\textwidth}
    \includegraphics[scale=0.9]{images/gitstash}
    \column{0.7\textwidth}
    \begin{itemize}
      \item \alert{Mostrar las diferencias}
          \mint{console}| $ git stash show <stash@{x}>|
      \item \alert{Limpiar el stash} \warnning
          \mint{console}| $ git stash clear|
      \item \alert{Sacar un guardado} \warnning
          \mint{console}| $ git stash drop <stash@{x}>|
    \end{itemize}
  \end{columns}
\end{frame}
